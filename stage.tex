\chapter{Les travaux effectuées et les apports du stage}
Au cours de ce stage, j'ai découvert le métier de consultant IT.
C'est un métier passionnant et varié.
Il requiert des capacités polyvalentes, un savoir-faire et de la flexibilité.

Les clients possédant chacun leur business, les problèmes qu'ils rencontrent ne sont jamais identique.
Il faut donc maitriser des connaissances dans plusieurs domaines et à plusieurs niveaux.
Avant d'être opérationnel, il faut posséder des connaissances théoriques et pratiques en infrastructure réseaux, en systèmes d'information.

Certaines opérations ne peuvent se faire qu'en dehors des heures de bureaux, il est nécessaire de travailler parfois plus tard ou le weekend. 
D'autres ne se font que chez le client, il faut se déplacer jusque chez ce dernier.

En cas d'urgence ou d'incident grave, il faut être prêt à partir pour résoudre le problème du client. 

Pour des raisons propres à l'entreprise, il ne m'a pas été possible de travailler directement avec les clients.
Mais en me plaçant dans l'open space avec les consultants, il m'a été donné l'opportunité d'observer le travail de ces derniers dans la résolution des incidents.

\section{Travaux effectuées}
Au cours de mon stage, mon travail a consisté essentiellement à réaliser l'architecture d'un laboratoire de test et à l'étude des connaissances relatives aux firewalls, passerelle SSL, tunnel VPN.

\subsection{Architecture du laboratoire de test}
Le laboratoire de la société permet aux employées de tester des nouvelles fonctionnalités, de simuler l'infrastructure d'un client afin de résoudre un problème, de configurer les machines à livrer à un client.
Sa disponibilité est capitale pour l'entreprise.

Le laboratoire possédait de vieux serveurs, qui n'étaient plus ou peu utilisés par manque de performance.
Les consultants travaillaient essentiellement avec cinq serveurs.
Ces serveurs faisait tourner un système d'exploitation à la fois.
Il n'était pas évident de créer une infrastructure complète.

Avec l'arrivée des nouveaux serveurs plus récent, l'utilisation du laboratoire est plus intéressante.
Il est désormais possible de créer une infrastructure complète en virtuel.

\subsubsection{Matériel mis à disposition}
Le laboratoire est assez complet d'un point de vue matériel. 
Il est composé d'une quinzaine de serveurs dont des NAS, de plusieurs switchs, de multiples routeurs et firewalls, et de PDU (Power Distribution Unit).
Je met une liste du matériel hardware
\begin{itemize}
	\item Serveurs HP Proliant DL360 et DL380
	\item Switch Cisco 3560G, 2960G, 2950
	\item Firewall Juniper NS25, SSG140
	\item Passerelle SSL SA2500
\end{itemize}
et une liste des systèmes d'exploitation et software sur lesquelles j'ai travaillé
\begin{itemize}
	\item Windows Server 2008R2, 2012 et 2012R2
	\item Windows Client Vista, 7, 8 Professional et Enterprise
	\item ESXi 5.5
	\item ScreenOS 6.3r12
	\item JunOS 8.1r1.1
	\item Exchange 2013
	\item Office 2013
\end{itemize}

\subsubsection{Design du laboratoire}
Dans un premier temps, un collègue m'a présenté le schéma logique du nouveau plan du laboratoire.
Sur base de ce plan, nous avons exploré l'ensemble des technologies à mettre en place.
Ce plan prévoyait la création de deux sous-réseau à l'intérieur du réseau de laboratoire.
Pour créer ces sous-réseaux, nous avons configuré deux firewalls NetScreen 25 de Juniper. 
Il prévoyait également des connexions iSCSI entre les serveurs et les disque virtuels de stockage sur les NAS.

Le laboratoire est composé de quatre racks, le rack 4 est réservé aux connexions vers la salle de production et vers l'open-space.
Nous avons décidé de placer les serveurs dans le rack 1 et la matériel réseau dans le rack 2.
Le rack 3 contient des Access Point pour le labo et accueille une solution Voice de Cisco depuis peu.

\subsubsection{(Dés)Installation des serveurs}
Nous avons commencé par enlever les anciens serveurs.
Ensuite, nous avons récupéré les nouveaux serveurs, que nous avons entreposé en attendant un plan détaillé de la localisation de ces derniers.

Pour établir la disposition exacte des serveurs, nous avons relevé les composants de tous les serveurs.
Nous avons pris note du nombre de cœurs physiques, de la quantité de RAM ainsi que de la disposition des slots de RAM.
Pour les disques durs, nous avons regardé au RAID formé, ainsi que la capacité de chaque disque.

Ensuite, nous avons classé les serveurs du plus puissant au moins performants. 
Les six serveurs les plus puissants sont configurés comme host de machine virtuelle. 
Quatre tournent sous Windows 2012R2 Datacenter avec le rôle Hyper-V et les deux autres tournent sous ESXi 5.5.
Parmi les autres serveurs, nous en avons choisi un pour faire tourner uniquement le service WSUS (Windows Server Update Services) sur un Windows Server 2012R2. 
Ce service permet de mettre à jour toutes les machines Windows qui s'y connectent.
Il télécharge les mises à jour depuis le serveur de Microsoft et est capable de les déployer sur les machines locales.
Cet outil est utile quand il y a beaucoup de machines sur le réseau à mettre à jour. 

\subsubsection{Connexion des serveurs}
Pour les serveurs, les manipulations étaient simples. 
Il nous suffisait de débrancher tous les câbles du serveur.
Ensuite de l'enlever des rails et d'enlever ces derniers du rack pour récupérer l'espace.
Pour placer les nouveaux, nous placions le rail en premier, puis le serveur dessus.
Le câblage des serveurs s'est fait un peu plus tard, car nous devions vérifier le nombre de connecteur réseau. 

En effet, nous avons crée des EtherChannel quand c'était possible.
Pour créer les EtherChannel, nous avons configuré le NIC Teaming sur les serveurs en mode LACP.
Nous avons aussi configuré les ports sur le switch afin de gérer les EtherChannel.
 
Nous avons besoin d'une grande bande passante entre nos serveurs et nos NAS (Network-Attached Storage) pour réduire les temps de transfert des VM (Virtual Machines).
Nous avons profité du fait que le switch principal a tous ses ports en gigabit. 
Ainsi, nous avons au minimum 1Gbps de vitesse de transfert et grâce au EtherChannel, nous montons jusqu'à 4Gbps.
Nous stockons les VMs sur les NAS, et chaque serveur possède une connexion iSCSI vers un iSCSI Target. 

L'avantage d'utiliser Windows 2012 R2 est que le NIC Teaming peut se faire via le Server Manager alors que dans les anciennes versions, il fallait utiliser le software du fabricant.


Pour la câblage, nous avons utilisé les bonnes pratiques en "cable management" :  
\begin{itemize}
\item Tous les câbles sont étiquetés aux deux extrémités
\item Utilisation d'un code couleur pour les différents types de câbles
\item Séparation physique entre les câbles Ethernet et les câbles d'alimentation
\item Emploi de câbles de bonnes longueurs
\end{itemize}

\subsubsection{Switch KVM}
Un switch KVM (Keyboard, video and mouse) est installé dans le rack.
Il nous permet de contrôler un serveur à distance comme si nous étions dessus en physique.
Il simule le clavier, l'écran et la souris sur le serveur. 

\subsubsection{Solution Voice}
Par la suite, nous avons reçu un solution complète de Voice.
Cette solution est composée d'un serveur, de six routeurs et six switchs.
Il permet de créer une infrastructure Voice d'une entreprise.

\subsubsection{Partie software}
Comme dit au point précédent, certains serveurs sont des hôtes pour des VMs.
J'ai donc installé des serveurs Windows 2012 R2, ainsi que des ESXi 5.5.

Les serveurs n'étant plus à jour, une mise à jour des drivers était nécessaire surtout pour les machines passant à un Windows Server 2012R2.
Les mises à jour ont été effectuées à l'aide d'une WebApp de HP.

Le laboratoire ne gère pas les domaines Active Directory, car son fonctionnement n'est pas compatible avec l'utilisation que l'on fait du labo. 
Les machines que nous installons, en virtuel ou physiquement, ont des durées de vie limitées.
Elles restent active entre une semaine et un mois.

\subsubsection{Plan d'adressage}
Sur base du classement, j'ai aussi défini le plan d'adressage des serveurs (voir Tab.\ref{tab:addIP} p.\pageref{tab:addIP}).
Même si la plupart des serveurs possèdent plusieurs cartes réseaux, ils n'ont qu'une seule adresse IP attribuée.
En effet, nous avons utilisé du NIC Teaming pour regrouper les interfaces physiques en un seul interface logique.
Il y a juste les serveurs ESXi qui ont deux adresses IP : la première sert au management et la deuxième permet les connexions iSCSI.
Cette séparation n'est pas nécessaire, mais elle est conseillée.
\begin{table}
\centering
\begin{tabular}{cc}
\toprule
Nom du serveur & Adresse IP \\
\midrule
HV-01 & 192.168.6.10 \\ 
HV-02 & 192.168.6.11 \\ 
ESXi-01 & 192.168.6.12 (Mgmt) - 192.168.6.21 (iSCSI) \\ 
ESXi-02 & 192.168.6.13 (Mgmt) - 192.168.6.20 (iSCSI) \\ 
HV-03 & 192.168.6.14 \\ 
HV-04 & 192.168.6.15 \\ 
WSUS & 192.168.6.16 \\ 
HP-NAS-01 & 192.168.6.17 \\ 
HP-NAS-02 & 192.168.6.18 \\ 
HP-NAS-03 & 192.168.6.19 \\
\bottomrule
\end{tabular}
\caption{Adressage des serveurs du laboratoire}
\label{tab:addIP}
\end{table}

Une fois les systèmes d'exploitation installés et les adresses configurées, tous les serveurs sont accessibles en remote. 

\subsection{Rôles des serveurs}
Nous avons installé les rôles Hyper-V sur les quatre serveurs dédiés, et nous avons configuré les iSCSI Initiator. 
Le role Hyper-V permet de manager des machines virtuelles. 
C'est une solution Windows, elle ne fonctionne qu'avec des VHD (Virtual Hard Disk). 

ESXi est la solution gratuite de virtualisation de VMWare. 
Pour manager les VMs, le client vSphere Hypervisor doit être installé sur une machine.

\subsection{Règlage du WSUS}
Le serveur WSUS fonctionnait correctement.
Mais pour le rendre accessible aux machines, nous avons dû modifier une valeur du registre windows.
Cette modification a pour but d'obliger la machine à contacter le serveur WSUS interne plutôt que le serveur Microsoft. 

\subsection{Templates Windows}
Il m'a été demandé de créer des \textit{templates} des systèmes d'exploitations.
Ces \textit{templates} ont pour objectif de fournir des systèmes pré-installés que nous pouvons déployer directement.

La création de ces derniers est simple. 
Il suffit de posséder une image du système d'exploitation.
Ensuite, nous démarrons la VM pour l'installation de base. 
Dès que le système est installé, nous avons appliqué toutes les mises à jour requises depuis le serveur WSUS. 
Finalement, nous avons exécuté la commande \texttt{sysprep} dans le \textit{cmd}.
Cette commande modifie des valeurs uniques créées par le système.
Par exemple, deux serveurs issus du même \textit{template} sans avoir lancé la commande \texttt{sysprep} ne peuvent pas être dans le même domaine, car ils possèdent des identifiants identiques. 

\section{Fait Marquant}
Lors des tests de l'infrastructure de mon TFE, nous sommes tombés sur un problème de connectivité entre une machine distante et un serveur Exchange interne via un tunnel VPN. 
Ce problème était en cours de résolution chez un client.
Le fait d'avoir pu recréer cet incident, nous a permis de fournir une solution au client concerné. 
Comme nous possédions une infrastructure similaire, nous avons pu mener un ensemble de test afin de déterminer le problème.
Finalement, nous avons trouvé une solution en interne, que nous avons pu transmettre au client et ainsi clôturer l'incident.

\section{Apports du stage}
J'ai appris énormément de chose pendant ces quatorze semaines de stage aussi bien d'un point de vue technique que humain.

\subsection{Compétences théorique acquises}
L'apprentissage des technologies passe par une phase d'étude, puis par la pratique.
J'ai passé un certain temps à étudier le fonctionnement de divers protocoles liés à la sécurité des données tels que IPSec et SSL/TLS.
De plus, j'ai du m'habituer au terme technique utilisé au sein de la société.

\subsection{Compétences techniques acquises}
Lors de ce stage, j'ai appris à installer des serveurs en suivant de bonnes pratiques, à designer un laboratoire en respectant les règles de "cable management", à configurer des firewalls, des switchs, des passerelles SSL. 
\subsubsection{Infrastrucutre Windows}
Une infrastructure professionnelle en Windows possède au minimum un serveur Active Directory (AD) avec le rôle Domain Controller (DC) et un serveur Exchange. 
L'utilisation d'Exchange requiert l'utilisation de certificat, il est donc nécessaire d'ajouter le rôle Certificate Services (CS) à l'AD.

La gestion des certificats est un peu complexe d'un premier abord.
Un collègue a pris le temps de m'expliquer le rôle d'une autorité de certification (CA) et comment créer une requête pour un serveur auprès du CA. 

Pour le stockage entre les serveurs et les NAS, la meilleur solution consiste à utiliser des disques iSCSI.
La nomenclature iSCSI est assez lourde, mais son fonctionnement est simple. 
Il suffit de créer un espace de stockage dynamique ou statique que l'on associe comme iSCSI Target. 
Ensuite, les clients, appelés iSCSI Initiator, se connectent aux disques cibles. 
Selon les autorisations, l'accès est accepté ou refusé.
L'avantage d'un disque iSCSI est qu'il est visible par le système d'exploitation de la même façon qu'un disque local. 

\subsubsection{Firewall Juniper}
Une notion essentielle de la gestion des firewalls Juniper est celle de Virtual Router.
Dans un firewall Juniper, les zones de confiances et les zones de non-confiances sont isolées l'une de l'autre.
C'est comme si on avait deux routeurs connectés l'un à l'autre dont l'un distribue les routes pour la zone de confiance et l'autre fait de même pour les zones de non-confiances.
Pour que du trafic passe entre les deux routeurs, les routes doivent être configurées manuellement.

En plus des routes, les firewalls se basent sur l'utilisation de Policies pour filtrer le trafic.
Les policies fonctionnent soit avec des adresses IP, soit avec des objets. 
Il est conseillé d'utiliser les objets, car ils masquent les adresses IP associées. 
Ainsi, lors d'un changement d'adressage, il suffit de modifier les propriétés de l'objet plutôt que d'aller modifier toutes les policies impactées.
\subsubsection{Passerelle SSL}
Une passerelle SSL gère les connexions distantes.
Elle dispose d'un ensemble de règles d'authentification et d'accès pour autoriser ou non une connexion distante.

\subsubsection{EtherChannel}
L'EtherChannel permet d'assembler d'un point de vue logique plusieurs ports d'un switch pour n'en former qu'un.
L'un des avantages est une plus grande bande passante, car chaque lien est indépendant de l'autre.
Ainsi, le système peut utiliser tous les liens pour l'envoi de données ou la réception ou faire un mix entre les deux en fonction de la charge. \\

Ces connaissances m'ont été utile pour mon travail de fin d'études. 

\subsection{Difficultés rencontrées}
Dans un premier temps, les problèmes étaient surtout d'un niveau théorique vu que je ne connaissais pas toutes les technologies à mettre en place. Avec le temps, ce genre de problème est devenu de plus en plus rare.
Par contre, les problèmes de configuration et de matériel sont devenus plus fréquents.

\subsection{La vie en société}
Travailler dans une PME a des avantages pour les relations humaines.
Il est en effet plus facile de discuter avec l'ensemble des employées.

Au début, il m'a été assigné un collègue qui servait de point de contact lorsque j'avais une question.
Mais le métier de consultant implique de se déplacer soit chez un client, soit dans un centre de formation, soit en urgence.
Il est arrivé que ce collègue part en milieu de journée ou qu'il soit absent une semaine complète pour donner un cours.
Au lieu de perdre mon temps, j'ai demandé des conseils aux autres collègues présent au bureau.
C'est ainsi que j'ai remarqué le niveau de compétence de l'ensemble des consultants.

L'ambiance au bureau était bonne, nous pouvions avoir des moments de rigolade, et aussi des moments de travail au calme.