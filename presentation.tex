\chapter{Le cadre du stage}
\section{Présentation de l'entreprise}
La société Matias Consulting Group (M.C.G.) est composée d'une dizaine de personnes. 
Elle est active dans les technologies de l'information et de la communication (TIC).

Le CEO est Grégorio Matias.
L'administratif est géré par Christine Warin (\textit{Management Assistant}).
La gestion des maintenances est planifiée par Dimitri Girboux (\textit{Service Delivery Manager}).
La partie commerciale de l'entreprise est confiée à Guillaume Verhaegen (\textit{Account Manager}).
Les formateurs sont entrainés et supervisés par Abdeltif Kamil (\textit{Training Delivery Manager}).

Les consultants IT sont certifiés auprès de plusieurs fabricants dont Cisco, Juniper, Microsoft et plus récemment Vasco.
Ils ont tous les certifications CCNA R\&S\footnote{Cisco Certified Network Associate : Routing \& Switching}, MCSA\footnote{Microsoft Certified Solutions Associate}.

De plus, ils sont tous formateurs soit dans des centres de formation, soit en entreprises.
Les cours donnés vont de la base des réseaux, avec un cours de TCP/IP, jusqu'à des notions avancées, avec un cours sur l'administration des réseaux en Windows 2012 R2. 
Ils donnent aussi des cours sur le Wireless et la Voice over IP (VoIP).

\subsection{Activité de l'entreprise}

Ses opérations portent sur trois points essentiels : 
\begin{itemize}
	\item L'audit et la conception d'infrastructures
	\item L'implémentation et la mise en œuvre de solutions
	\item La formation, le support et la maintenance
\end{itemize}

\subsection*{Audit et conception d'infrastructure}
Avant de vendre le matériel à un client, il est obligatoire d'évaluer ses besoins réels.
Une étude complète des besoins permet de fournir le matériel adapté au business du client.


\subsection*{Implémentation et mise en œuvre de solutions}
L'entreprise gère l'installation du matériel chez le client, si ce dernier le demande.
Elle possède les compétences quant à l'installation et la configuration du matériel informatique.
Par matériel informatique, on parle principalement de poste de travail, de serveurs, de firewall, ...
Elle se charge de commander le matériel auprès des fournisseurs.
En règle générale, les commandes sont livrées chez M.C.G..
Lorsque la commande est reçue, nous livrons au client selon ses disponibilités.
Le fait de livrer à la société permet, d'une part de vérifier le fonctionnement du matériel, d'autre part de réaliser une configuration avant la livraison.

\subsection*{Formation}
La formation fait partie intégrante de la politique de la société.
Elle dispense des formations dans les centres de formation en Wallonie et à Bruxelles. 
Elle forme aussi les employés des entreprises sur base des besoins de cette dernière.

\subsection*{Support et maintenance}
La société possède une hotline pour les clients en cas de problème. 
Pendant les heures de bureau, il y a toujours quelqu'un pour répondre à cette ligne et ainsi aider le client.

Il en va de même pour la maintenance, un appareil de monitoring est allumé en permanence et un employé est derrière pour signaler la moindre avarie sur les appareils monitorés.

M.C.G. peut fournir des consultants en société pour les entreprises qui ont des besoins spécifiques. 
Dans ce cadre, il est détaché pour une mission/projet de plus ou moins grande échéance.
Pour autant, il reste encadré et supervisé par MCG.
