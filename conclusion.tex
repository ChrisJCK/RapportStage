\chapter*{Conclusion}
\addcontentsline{toc}{chapter}{Conclusion}
Ce stage a été une excellente opportunité pour découvrir le métier de consultant.
Il m'a permis de ma faire une idée du niveau de connaissance technique requis pour travailler dans ce domaine.
Il m'a aussi fait découvrir la formation continue.
J'ai apprécié le fait que les consultants soient aussi des formateurs.\\

Ma mission principale était la conception du laboratoire de test.
Le laboratoire est opérationnel à un détail près.
Il manque un des NAS initialement prévu, car il n'a pas été livré durant la période de mon stage. \\

Durant ce stage, j'ai appris un ensemble de technologies et de pratiques qui m'aideront plus tard.
J'ai appris essentiellement sur les systèmes d'exploitation Windows et les équipements réseaux de Cisco et Juniper. 
Pour ce qui concerne la méthodologie de travail, j'ai appris à faire des recherches avant de foncer tête baissée dans un projet.
Pour éviter des problèmes, il est nécessaire de planifier et de prévoir les différents étapes du projet.

Bien que le respect d'un planning me fait défaut, je pense qu'il est important de planifier le plus possible les étapes. 
Avec la pratique, cette lacune devrait disparaître.\\

Avec l'expérience et les connaissances acquises, j'aimerais bien continuer dans cette voie, c'est-à-dire les réseaux et les systèmes.
Il me reste juste à résoudre le choix entre consultant et interne.
Ce choix n'est pas évident, car les deux sont intéressants.

Le métier de consultant est exigeant sur les connaissances à maîtriser.
Il demande une certaine flexibilité sur les horaires.
De plus, les certifications sont un vrai plus.
Il est donc nécessaire de rester à jour et de suivre des formations régulièrement.

Le métier d'informaticien interne (administrateur réseau) est tout aussi exigeant mais sur d'autres points.
En règle générale, les horaires sont fixes et les compétences sont liées au matériel de la société.